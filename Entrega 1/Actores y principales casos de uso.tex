% Plantilla para documentos LaTeX para enunciados
% Por Pedro Pablo Aste Kompen - ppaste@uc.cl
% Licencia Creative Commons BY-NC-SA 3.0
% http://creativecommons.org/licenses/by-nc-sa/3.0/

\documentclass[12pt]{article}

% Margen de 1 pulgada por lado
\usepackage{fullpage}
% Incluye gráficas
\usepackage{graphicx}
% Packages para matemáticas, por la American Mathematical Society
\usepackage{amssymb}
\usepackage{amsmath}
% Desactivar hyphenation
\usepackage[none]{hyphenat}
% Saltar entre párrafos - sin sangrías
\usepackage{parskip}
% Español y UTF-8
\usepackage[spanish]{babel}
\usepackage[utf8]{inputenc}
% Links en el documento
\usepackage{hyperref}

\begin{document}

\begin{minipage}{2.3cm}
\includegraphics[width=2cm]{img/logo.jpg}
\vspace{0.5cm} % Altura de la corona del logo, así el texto queda alineado verticalmente con el círculo del logo.
\end{minipage}
\begin{minipage}{\linewidth}
\textsc{\raggedright \footnotesize
Pontificia Universidad Católica de Chile \\
Departamento de Ciencias de la Computación \\
IIC2513 - Tecnologías y Aplicaciones Web \\
Proyecto: Save me Sensei!\\
Grupo: Rupie\\
Germán Contreras\\
Óscar Estay\\} % REEMPLAZAR EL CURSO
% INSERTAR NOMBRE(S)
\end{minipage}


% Titulo
\begin{center}
\vspace{0.5cm}
{\huge\bf Entrega 1}\\
\vspace{0.2cm}
16 de Septiembre de 2015\\
%\vspace{0.2cm}
%\footnotesize{Ayudante: Pedro Aste - ppaste@uc.cl}
\vspace{0.2cm}
\rule{\textwidth}{0.1mm}
\end{center}

% Cuerpo del documento ejemplo

\section*{Actores y principales casos de uso}

Los actores pueden ser divididos en:
\begin{itemize}
  \item{Usuarios inscritos: Los usuarios inscritos tienen un perfil en la base
  de datos de la aplicación. Contienen un username que los identifica, su mail,
  una password, y un perfil. Dentro del perfil, el usuario puede agregar su
  nombre a la cuenta (que por default será igual a su username), una foto,
  configurar su horario de disponibilidad, agregar un rango de dinero que está
  dispuesto a pagar o cobrar por sus clases y agregar materias sobre las que
  quiere aprender o enseñar. Además, el perfil mostrará la cantidad de clases
  recibidas o impartidas, una calificación como alumno y como profesor, dada por
  los profesores y alumnos que haya tenido, y también tendrá un muro, donde los
  demás usuarios inscritos podrán escribir mensajes.\\
  Al abrir la aplicación, los usuarios inscritos serán recibidos por un portal,
  que mostrará los últimos avisos publicados, algunos random y un cuadro en el
  que el usuario podrá publicar un aviso. También tendrá un panel que le dará
  acceso a su perfil, al foro, a la configuración de su información y a una
  navegación por los avisos. Contendrá un buscador de avisos y un buscador de
  personas\\
  Tendrá autorización para publicar preguntas o algún otro tema y comentar en el
  foro. También podrá calificar usuarios con los que haya tenido interacciones y
  publicar en el muro de los mismos.\\
  }
  \item{Usuarios no inscritos: Los usuarios no inscritos no podrán tener
  interacciones con los demás usuarios inscritos, pero si podrán navegar por la
  aplicación. Podrán ver los mensajes del foro, los perfiles de los usuarios y
  los avisos publicados.\\}
  \item{Administradores: Usuarios inscritos pero con el poder de hacer tareas
  administrativas, como borrar temas del foro, de algún muro, cerrar terminar el
  foro }

\end{itemize}

% Fin del documento
\end{document}
